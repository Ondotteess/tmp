\documentclass{article}
\usepackage[utf8x]{inputenc}
\usepackage[T2A]{fontenc}



\title{}
\author{}
\date{}

\begin{document}

\maketitle

\section{Решение квадратного уравнения}
\textit{Задача:} решить уравнение 2\textit{x}^2+5x-12=0.
\par \textit{Решение.} Это квадратное уравнение, общий вид которого:
$$ ax^2+bx+c=0. $$
В нашем случае a = 2, b = 5, c = --12.
\par Сначала необходимо вычислить дискриминант уравнения:
$$D = b^2-4ac=(5)^2-4\cdot2\cdot(-12)=121$$
Так как дискриминант является положительным (D > 0), это уравнение имеет два корня, вычисляемые по формуле:
$$x_{1,2}=\frac{-b\pm\sqrt{D}}{2a}=\frac{-5\pm11}{4}.$$
Таким образом, \textit{x}_1=\frac{-16}{4}=-4,  \textit{x}_2=\frac{6}{4}=\frac{3}{2}.
\par \textit{Ответ: x}_1 = -4, \textit{x}_2=\frac{3}{2}

\end{document}
